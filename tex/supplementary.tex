
% PLEASE USE THE DRAFT OPTION TO SUBMIT YOUR PAPERS.
\documentclass[draft,jgrga]{agutexSI2019}

 \usepackage{graphicx}
\usepackage{makecell}%
\usepackage{booktabs}
%  Uncomment the following command to allow illustrations to print
%   when using Draft:
 \setkeys{Gin}{draft=false}



\begin{document}


\title{Supporting Information for ...}

\authors{}


\affiliation{1}{EOST/ITES, Strasbourg University/CNRS, 5 rue René Descartes, 67000 Strasbourg, France}








%% ------------------------------------------------------------------------ %%
%
%  BEGIN ARTICLE
%
%% ------------------------------------------------------------------------ %%



\begin{article}

\begin{enumerate}
    \item Bibliography of fluid-driven seismicity migration processes
    \item model presentation: 3DEC + semi-analytical solutions [comparison/benchmark] (summary) - focus on soft regime with normal stress pulse propagation [discussion of the stiff/soft transition is the focus of the RMRE paper - not really discussed here] - shear stress pulse -  Coulomb failure criterion / Slip weakening law - Rate and State friction (earthquake initiation) - quasi-dynamic approximation 
    \item square-root propagation of normal stress pulse for pressure boundary condition - how far can go the stress pulse?
    \item departure from square-root propagation for imposed volume condition - effect of the injection rate [constant] on the stiff/soft transition
    \item arrest phases
    \item discussion: application to seismicity migration interpretation
    \item perspective: gravity-driven pulse migration
\end{enumerate}

% \begin{enumerate}
%     \item Bibliography
%     \begin{itemize}
%         \item nonlinear diffusion and its effect of induced seismicity
%         \item review on fluid-driven induced seismicity migration
%     \end{itemize}
%     \item model presentation: 3DEC + semi-analytical +  numerical? (is easier to analyze bigger fractures): 
%     \begin{itemize}
%         \item boundary condition - various constant injection rates;
%         \item  initial conditions: soft fracture from last paper; 
%         \item fracture deformation model - only constant normal stiffness model?
%     \end{itemize}
%     \item Results
%     \begin{itemize}
%         \item Pressure, aperture and stress profiles
%         \item Migration of stress peak along the fracture (time-behavior)
%         \item Consider bigger fracture (km scale)
%         \item Simulate induced seismicity with Mohr-Coulomb failure law using 3DEC
%     \end{itemize}
% \end{enumerate}

[[@GoebelBrodsky2018]] by analyzing the fluid injection induced seismicity from multiple wells have identified two different groups of spatial decay of seismicity event density away from the injection well: (i) is a seismic event density that stays constant for some distance, and then within 1 km distance abruptly decays, (ii) event density starts to steadily decay right from the beginning until over 5 km. To quantitatively describe this they fit of seismic event density equation in function of distance with an exponent $\gamma$ that describes the "abruptness" of the spatial decay at larger distances. This analysis also showed a difference between the two groups: steady decay with $\gamma=1.5-3.1$ and abrupt decay with $\gamma=4.3-5.9$. Furthermore, [[@GoebelBrodsky2018]] studied the seismicity migration at these fluid injection sites. They found that at majority of "abrupt decay" sites, the seismicity migrates with square-root of time. On the other hand, "steady decay" sites showed mostly either linear or no migration trend. 

They found the reservoir rock type to be the most important parameter causing different behavior in the spatial decay. For abrupt decay sites fluid was injected in the basement rock (crystalline rocks), and sites with steady decay were located in sedimentary rocks (except the Basel case). They propose that the distinct behavior of spatial seismicity migration for different rock types may be due to fluid pressure induced poro-elastic stresses. In sedimentary rocks poro-elastic stresses dominate which are induced beyond the pressure disturbance reach (Segall and Lu, 2015). Therefore, due to elastic stress effects the seismicity decays steadily (power-law decay). Whereas, in basement rocks, fluid pressure migration dominates, where seismicity density is almost constant with distance close to the injection well at it decays abruptly close to the maximum seismicity distance. (Goebel et al., 2017) used the poroelasticity effects to explained the seismicity that was triggered 40km far from the injection point. With their numerical and semi-analytical models they could reproduce the coulomb stress changes that were induced by fluid induced elastic response that can reach such high distances far from the injection point.

Understanding the mechanisms behind injection induced seismicity motivated Guglielmi et al., (2015) to carry out in-situ injection experiment in the south of France. During the fluid injection the fault-permeability increase by 14 fold, suggested to be mainly induced by the slip along the fault. Moreover, the seismicity start much later (700 s) than the start of the slip of the fault. This observation highlights the existence of aseismic slip during fluid injection. The aseismic slip is attributed to the rate-strengthening nature of the fluid filled region of the fault. And a relatively simple model that matches the experimental results show that seismicity starts when the radius of slip zone exceeds the radius of fluid pressurized zone.

(Bhattacharya et al., 2019) did a numerical simulation with a model properties and friction laws that were inferred from the in-situ fluid injection experiment of Guglielmi et al., 2015. Their main finding is that for a critically stressed fault the aseismic slip outpaces the fluid pressure front. The slip front followed a square-root-of-time law with a constant proportionality that depends on the criticality of the fault.

The observation of aseismic slip in fluid injection sites (e.g. Guglielmi et al., 2015) and the slip front not being equal to the pressure front (Guglielmi et al., 2015; Bhattacharya et al., 2019) motivated Barros et al., (2021) to do a numerical model fluid injection that accounts for aseismic slip. Their model finds that in initially critically stressed faults during fluid injection, the aseismic slip eventually propagates ahead of the pressure front and migrates with constant velocity. Therefore, seismicity front at first follows the pressure front but later it follows the aseismic slip front. (De Barros et al., 2021) tried to back their modeling results with the observations of seismicity from some injection sites, especially the Basel site. And they found that seismic front migration behavior changes with time. Particularly initially the seismicity migration follows close to square root of time behavior. Whereas, the second half of the seismicity sequence, follows quasi-linear seismicity migration. But this behavior is not seen Soultz hydraulic stimulation case.

\clearpage


\end{article}

\begin{figure}
    \centering
    \includegraphics[width=1\linewidth]{figures_supp/front-distance-time-1bar-const-p.png}
    \caption{1 bar pressure front position over time for rigid and soft fracture with constant overpressure applied at the injector. Square root of time propagation for both rigidity cases rigid fracture having faster pressure propagation.}
\end{figure}


\begin{figure}
    \centering
    \includegraphics[width=1\linewidth]{figures_supp/aperture.png}
    \caption{Aperture profiles for various constant injection rate applied}
    \label{fig:placeholder}
\end{figure}

\begin{figure}
    \centering
    \includegraphics[width=1\linewidth]{figures_supp/pressure.png}
    \caption{Pressure profiles for various constant injection rate applied}
    \label{fig:placeholder}
\end{figure}

\begin{figure}
    \centering
    \includegraphics[width=1\linewidth]{figures_supp/stress-profiles.png}
    \caption{Stress profiles for various constant injection rate applied}
    \label{fig:placeholder}
\end{figure}

\begin{figure}
    \centering
    \includegraphics[width=1\linewidth]{figures_supp/stress-migration.png}
    \caption{Tensile stress peak position in function of time obtained from numerical solution together with power law fit for various constant injection rate applied}
    \label{fig:placeholder}
\end{figure}

\begin{figure}
    \centering
    \includegraphics[width=1\linewidth]{figures_supp/comparison-rittershoffen.pdf}
    \caption{comparison of our model for normal pulse migration with Rittershoffen seismicity migration}
    \label{fig:placeholder}
\end{figure}


% \begin{table}
% \begin{adjustwidth}{-3cm}{-1cm}
%     \centering
% \caption{Fluid induced seismicity models with their associated mechanisms and results.}
% \label{tab:placeholder}
%     \begin{tabular}{|p{0.2\linewidth}|p{0.2\linewidth}|p{0.2\linewidth}|p{0.2\linewidth}|p{0.2\linewidth}|}\hline
%          Source&  Site& Mechanisms &Model &Main result\\\hline
%          \cite{GuglielmiEtAl2015b}&  France (in-situ experiment)&  Shear dilation; Aseismic slip& Circular crack with uniform pressure field&Seismicity begins when crack radius extends beyod the fluid pressure region\\\hline
%          \cite{GoebelEtAl2017}&  Oklahoma, USA&  Poroelastic stresses& Linear pressure diffusion; Poroelastic stresses&Poroelastic stress change dominate in the far field and induce significant Coulomb stress change\\\hline
%          \cite{GalisEtAl2017}&  Generic study&  Self-arrested rupture& Linear fluid pressure diffusion 3D dynamic rupture on a slip-weaking fault; Griffith crack with a point load stress drop&Fluid-induced ruptures are self-arrested; pressure distribution is irrelevant, only its spatial extent matters;  fault parameters control the rupture\\\hline
%          \cite{BhattacharyaViesca2019}&  France (in-situ experiment \cite{GuglielmiEtAl2015b})&  Aseismic slip& Linear pressure diffusion with step increase of permeability, elliptical shear rupture due to decrease of frictional strength&In critically stressed faults, aseismic slip front propagates faster than the diffusive fluid pressure front\\\hline
%          \cite{DeBarrosEtAl2021}&  Generic study (\cite{Wynants-MorelEtAl2020} model) compared with data&  Aseismic slip& Hydromechanically coupled 3DEC code with highly permeable, rigid and slip-weaking fault&Initially seismicity diffuses, later on, if the fault is critically stressed, aseismic slip induced seismicity accelerates\\\hline
%          \cite{DanreEtAl2024}&  Generic study compared with data&  Aseismic slip& Circular crack rate-and-state model based on injected volume loaded by a point pressure for hydraulically conductive fault zone&The model  explains the observations. Seismicity migrates with square root of fluid volume\\\hline
%          \cite{YeoEtAl2020}&  Pohang, Korea&  Coulomb stress transfer& &Pore-pressure initiates seismicity and Coulomb stress transfer (slip induced stress change) promote continued seismicity\\ \hline
%          \cite{ShapiroDinske2009}&  Generic model compare with data&  Nonlinear pressure diffusion& &Opening of pre-existing fractures is the dominant mechanism controlling the dynamics of induced seismicity.\\ \hline
%     \end{tabular}
    
% \end{adjustwidth}
% \end{table}



\clearpage




 \bibliography{references} 

\end{document}
