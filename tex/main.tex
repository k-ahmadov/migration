%% To submit your paper:
\documentclass[draft]{agujournal2019}
\usepackage{lmodern}
\usepackage{url} %this package should fix any errors with URLs in refs.
\usepackage{makecell}
\usepackage{booktabs}
\usepackage{lineno}
\linenumbers
\usepackage{multirow}
\usepackage[colorinlistoftodos]{todonotes}
% \draftfalse
\journalname{JGR: Solid Earth}
% As of 2018 we recommend use of the TrackChanges package to mark revisions.
% complete documentation is here: http://trackchanges.sourceforge.net

\begin{document}

%  TITLE
\title{Normal stress pulse during fluid-driven seismicity migration}

%  AUTHORS AND AFFILIATIONS
\authors{K. Ahmadov\affil{1}, J. Schmittbuhl\affil{1}, T.G.G Candela\affil{2}}
\affiliation{1}{EOST/ITES, Strasbourg University/CNRS, 5 rue René Descartes, 67000 Strasbourg, France}
\affiliation{2}{Geological Survey of the Netherlands, TNO, Utrecht, the Netherlands}
\correspondingauthor{}{}

% KEY POINTS
\begin{keypoints}
	\item
	\item
	\item
\end{keypoints}

%  ABSTRACT
\begin{abstract}

	\begin{enumerate}
		\item Bibliography
		      \begin{itemize}
			      \item Review of spatio-temporal seismicity migration data and the associated migration models
			      \item Review of models of fluid injection and the proposed physical mechanisms of induced seismicity
		      \end{itemize}
	\end{enumerate}
\end{abstract}

\section*{Plain Language Summary}

\section*{Conflict of Interest}
The authors declare no conflicts of interest relevant to this study.

%%%%%%%%%%%%%%%%%%%%%%%%%%%%%%%%%%%%%%%%%
%
%  BODY TEXT
%
%%%%%%%%%%%%%%%%%%%%%%%%%%%%%%%%%%%%%%%%%

\section{Introduction}

\todo[inline]{TL;DR Fluid injection induced seismicity generally migrates and follows a power-law relationship in time}

Spatial migration is a common feature of earthquake sequences induced by fluid injection.  Observations show that the hypocenters of induced-earthquakes migrate away from the injection point and exhibit a temporal dependence \cite{ShapiroEtAl2002, ChenEtAl2012, DeBarrosEtAl2021, DanreEtAl2024}. The spatiotemporal migration of these earthquakes is commonly analyzed in distance-time plots where the distance of each seismic event relative to the origin position is linked with the elapsed time. An important feature that often emerges from these plots is the seismic front, generally defined as the nearly farthest extent of seismicity at a given time \cite{GoebelBrodsky2018, DanreEtAl2024}. The evolution of the seismic front can be described by the following power-law relationship in time:
\begin{equation}\label{seismic-front-vs-time}
	d(t) \propto t^{\alpha}
\end{equation}
where $d$ is the distance of seismic events relative to the origin position, $t$ is time, and $\alpha$ is the exponent of time.

\todo[inline]{TL;DR Reported exponents of time $\alpha$ range between 0 and 1. Often, $\alpha$ varies between injection sites and among different studies of the same site (Table 1)}

Early studies that identified and analyzed the power-law scaling in seismic front migration found that a square-root-of-time relationship best fits the observed seismic front evolution \cite{ShapiroEtAl1997, ShapiroEtAl2002}. Since then, a growing body of literature has examined the spatial migration of the seismic front and searched for a power-law relationship in time (Eq.~\ref{seismic-front-vs-time}) that describes its evolution. These studies have investigated seismicity migration across multiple sites worldwide \cite{GoebelBrodsky2018,DeBarrosEtAl2021}, as well as at individual sites \cite{ShapiroEtAl2002,LenglineEtAl2017,DanreEtAl2024}.  Table~\ref{tab:migration-observations} summarizes exponents of time $\alpha$ for seismic front migration reported for several injection sites. At many sites, the inferred exponents deviate from the classical square-root-of-time scaling and range between 0 and 1. Four general migration patterns emerge: square-root-of-time, sub-1/2 exponents, quasi-linear or linear migration, and the lack of observable migration ($\alpha=0$). The reported time exponents that describe the seismic front evolution varies across injection sites. Notably, $\alpha$ also varies among different studies of the same site.  This is evident in the case of Soultz 1993 fluid injection experiment, for which \citeA{GoebelBrodsky2018} identify a square-root-of-time relationship as the best fit, \citeA{DeBarrosEtAl2021} estimate $\alpha$ to be approximately 0.65, and \citeA{SaezUribe2023, DanreEtAl2024} find linear pattern of migration in time to fit the observations well, except at early times. Another example of heterogeneity in exponents of time reported by different authors for the same dataset is the Basel injection site (Table~\ref{tab:migration-observations}). In conclusion, the literature of the analysis of seismicity migration data shows that the seismic front evolution does not follow a single relationship in time, instead it changes with different injection sites and among different researchers for the same site.

\begin{table}[htbp]
	\centering
	\caption{Observed seismicity migration behavior at various injection sites, including the migration model type and associated time exponent, as reported by different studies.}\label{tab:migration-observations}
	\begin{tabular}{llc}
		\toprule
		Site                              & Source                   & Time exponent, $\alpha$ \\
		\midrule
		\multirow{5}{*}{Soultz 1993}      & \cite{ShapiroEtAl2002}   & 1/2                     \\
		                                  & \cite{GoebelBrodsky2018} & 1/2                     \\
		                                  & \cite{DeBarrosEtAl2021}  & 0.65                    \\
		                                  & \cite{SaezUribe2023}     & 1                       \\
		                                  & \cite{DanreEtAl2024}     & 1                       \\
		\midrule
		\multirow{4}{*}{Basel 2006}       & \cite{JohannEtAl2016}    & 0.49                    \\
		                                  & \cite{GoebelBrodsky2018} & 1                       \\
		                                  & \cite{DeBarrosEtAl2021}  & 0.7                     \\
		                                  & \cite{DanreEtAl2024}     & 1                       \\
		\midrule
		\multirow{2}{*}{Cooper 2003}      & \cite{GoebelBrodsky2018} & 1/2                     \\
		                                  & \cite{JohannEtAl2016}    & 0.28                    \\
		Cooper 2012                       & \cite{DeBarrosEtAl2021}  & 0.5                     \\
		\midrule
		\multirow{2}{*}{Fenton Hill 1983} & \cite{ShapiroEtAl2002}   & 1/2                     \\
		                                  & \cite{GoebelBrodsky2018} & 1/2                     \\
		\midrule
		Rittershoffen 2013                & \cite{LenglineEtAl2017}  & 1                       \\
		Rittershoffen 2016-2024           & \cite{LenglineEtAl2025}  & 1/2                     \\
		\midrule
		KTB 1997                          & \cite{ShapiroEtAl1997}   & 1/2                     \\
		Barnett Shale                     & \cite{ShapiroDinske2009} & 1/3                     \\
		St. Gallen                        & \cite{GoebelBrodsky2018} & 1                       \\
		Paralana, Australia               & \cite{GoebelBrodsky2018} & 1                       \\
		Rustrel, France                   & \cite{GoebelBrodsky2018} & 0$^1$                   \\
		Landau                            & \cite{GoebelBrodsky2018} & 0                       \\
		Paradox, Colorado                 & \cite{GoebelBrodsky2018} & 1                       \\
		\bottomrule
		\multicolumn{2}{l}{$^1$ Seismic front lacks temporal dependence.}
	\end{tabular}
\end{table}

\todo[inline]{TL;DR Differences in fitting methods, spatial reference choices and seismicity front definition can lead to different interpretations of seismicity migration.}

Probably, the major reason behind the dissimilarity in the reported exponents of time $\alpha$ among different authors for the same injection site is the difference in the methodology of the analysis of seismicity migration data. The methodological choices that may differ between authors are: the spatial origin of the distance-time plot, the definition of the seismic front, and the fitting of the power-law relationship to the (Eq.~\ref{seismic-front-vs-time}) seismic front. The difference in methodological choices is exemplified in the Soultz injection site, where the method to fit the power law relation (Eq.~\ref{seismic-front-vs-time}) to the envelope of the seismic cloud differs amid different studies. For this case, to find the best fit relation, \citeA{GoebelBrodsky2018} consider only three discrete models—no migration, square-root-of-time, and linear—whereas \citeA{DeBarrosEtAl2021} fit a power-law function with a best-fit prefactor and time exponent to the seismicity front. Some studies, such as \citeA{ShapiroEtAl2002, SaezUribe2023, DanreEtAl2024}, derive the front evolution from theoretical or numerical models and then fit these predictions to the seismicity migration envelope.  As a consequence, the inferred $\alpha$ for Soultz case differs with the fitting approach (Table~\ref{tab:migration-observations}).  Another factor that may influence the inferred seismicity migration pattern is the choice of the spatial origin, which is inherently subjective \cite{DeBarrosEtAl2021}. For instance, \citeA{DanreEtAl2024} define the origin as the median location of the first ten events, \citeA{ShapiroEtAl1997} use the injection well as the origin, and \citeA{LenglineEtAl2017} select a point along the borehole trajectory at the depth of the first earthquake. Last but not least, the definition of the seismicity front may also change between studies that analyze the same injection site. In the case of the Basel site, \citeA{JohannEtAl2016} defines the seismicity front using an algorithm based on the farthest triggered events, whereas \citeA{DanreEtAl2024} choses the seismicity front to be in between 70th and 80th percentile of farthest events.  In summary, the fitting methodology, the choice of the origin, and the definition of the seismicity front can affect the analysis of seismicity migration.

\todo[inline]{Transition to mechanisms for observed exponents of time (Table 2)}

Researchers have proposed several mechanisms associated with fluid-induced seismicity which attempts to explain the different expontents of time obtained for those injection sites. Table~\ref{tab:induced-seismicity-mechanisms} presents these mechanisms which includes linear and nonlinear pressure diffusion, aseismic slip, poroelastic effects, and Coulomb stress transfer. In the following, we briefly review the mechanisms that previous studies propose to explain each observed seismicity migration behavior.

\todo[inline]{TL;DR Linear diffusion is a classical mechanism and its associated with the square-root-of-time migration}

The fluid overpressure is a classical mechanism that is proposed to induce seismicity during fluid injection operations. According to the Mohr--Coulomb failure criterion, an increase in fluid pressure decreases the effective stress acting on a fault, which brings it closer to failure \cite{king1959role}. In fractured rock systems, the fluid injected at a given position propagates according to the diffusion law. The pressure perturbation front derived from the linear diffusion equation propagates proportionally to the square-root-of-time \cite{MurphyEtAl2004}. Researchers have observed such diffusive behavior in numerous cases of fluid injection-induced seismicity sequences where the seismicity front follows a square-root-of-time scaling (e.g., \citeA{ShapiroEtAl2002}). This pattern of seismicity migration suggests that the injection-induced seismicity is mainly controlled by the fluid overpressure and its spatio-temporal evolution.

\todo[inline]{TL;DR The lack of observable induced seismicity migration ($\alpha=0$) may be the result of far-reaching poroelastic stress changes}

In addition to the diffusive migration of seimicity, observations show that, at some injection sites, the seismicity lacks a clear migrational pattern (Table~\ref{tab:migration-observations}). \citeA{GoebelBrodsky2018} suggest that injection-induced seismicity sequences lacking observable migration may result from far-reaching poroelastic stress changes. In particular, \citeA{GoebelEtAl2017} documented seismic events triggered far from the injection zone during wastewater disposal in Oklahoma, USA\@. Numerical analyses of these sequences indicate that stress perturbations extend beyond the fluid pressure front and dominate the far-field response.

\todo[inline]{TL;DR Seismicity that migrates with cubic-root-of-time is linked with nonlinear diffusion, nonetheless, such process may also result in square-root-of-time migration}

Injection-induced seismicity in the Barnett Shale and the Cooper Basin exhibits sub-diffusive migration characterized by time exponents $\alpha$ smaller than 1/2. Studies attribute these observations to nonlinear diffusion processes \cite{ShapiroDinske2009, JohannEtAl2016}. The nonlinear diffusion of fluid pressure during injection arises when the hydraulic transport properties of the medium, such as permeability and diffusivity, are pressure-dependent. In particular, \citeA{ShapiroDinske2009} showed that accounting for pressure-dependent diffusivity leads to a nonlinear diffusion equation, which in turn may alter the time scaling of the pressure front migration. They showed that the resulting scaling depends on the injection boundary condition, the dimensionality of fluid flow, and the degree of nonlinearity of the governing equation. Under conditions of strong nonlinearity and spherically symmetric (3D) diffusion, the model predicts $\alpha=1/3$, consistent with seismicity migration pattern observed in the Barnett Shale (Table~\ref{tab:migration-observations}).  However, observations from other injection sites indicate that nonlinear diffusion does not universally lead to sub-diffusive migration. In another study on the effect of nonlinear diffusion, \citeA{HummelMller2009} showed that, when constant injection pressure is applied, increasing the degree of nonlinearity does not have an effect on the seismicity front evolution. Instead, it primarily increases the seismic event density near the seismicity front. This behavior appears at the Fenton Hill injection site, where seismicity exhibits diffusive front migration ($\alpha=1 / 2$) but shows a pronounced concentration of events near the front. These observations indicate that seismicity migrating with cubic-root or square-root-of-time may both be linked with nonlinear diffusion process.

Several injection induced seismicity sequences exhibit quasi-linear seismicity migration (Table~\ref{tab:migration-observations}). Such constant-velocity seismicity migration is characteristic of natural earthquake swarms where the seismicity is mainly driven by slow aseismic slip \cite{DeBarrosEtAl2021, DanreEtAl2024}. This fact coupled with the numerous observations of aseismic slip during field experiments of fluid injection have motivated researchers to investigate its role in injection-induced seismicity \cite{GuglielmiEtAl2015b, BhattacharyaViesca2019, DeBarrosEtAl2021}. \citeA{BhattacharyaViesca2019} developed a numerical model using physical parameters inferred from an \textit{in-situ} fluid injection experiment conducted by \citeA{GuglielmiEtAl2015b}. Their simulations showed that, on critically stressed faults, the rupture front can propagate beyond the region of significant fluid overpressure, thereby modifying the apparent diffusivity of induced microseismicity inferred from distance–time plots. Similarly, \citeA{DeBarrosEtAl2021}, using the Distinct Element Method (DEM), demonstrated that on initially highly critical faults, accelerated aseismic slip can drive the seismicity front ahead of the fluid pressure front, leading to constant-velocity migration or even a migration with increasing velocity. Consistent with these findings, a theoretical fracture mechanics model that incorporates slow slip predicts linear seismicity migration at the Basel and Soultz injection sites \cite{DanreEtAl2024}. The model represents slow (aseismic) slip as a quasi-static crack and fluid pressure distribution as a point force. Together, these results suggest that aseismic slip governs the quasi-linear migration of seismicity observed at some fluid injection sites.

\todo[inline]{to be improved}
Aseismic slip, as an additional mechanism, can explain the quasi-linear seismicity migration observed at several injection sites. However, this mechanism, along with other mechanisms discussed previously, introduce additional complexity into the problem. In particular, these mechanisms introduce extra physical parameters and governing laws that remain poorly constrained. Moreover, they require assumptions about medium properties that are not clearly justified. For example, considering aseismic slip during fluid injection requires knowledge of the fault friction law and its associated parameters, which are generally poorly known. In addition, aseismic slip as a mechanism for fluid-induced seismicity requires the fault to be initially critically stressed \cite{BhattacharyaViesca2019, DeBarrosEtAl2021}, which may not hold at some injection sites. In conclusion, although mechanisms other than fluid diffusion can explain the previously observed ``non-diffusional'' features in the data, they introduce poorly constrained variables and additional degrees of freedom into the model.

\begin{table}
	\centering
	\caption{Fluid induced seismicity models with their associated mechanisms and results}\label{tab:induced-seismicity-mechanisms}
	\begin{tabular}{lllc}
		\toprule
		Mechanisms                           & Source                     & Site            & $\alpha$             \\
		\midrule
		\multirow{3}{*}{Linear diffusion}    & \cite{HsiehBredehoeft1981} & Denver          & \multirow{3}{*}{1/2} \\
		                                     & \cite{ShapiroEtAl1997}     & KTB, Germany                           \\
		                                     & \cite{ShapiroEtAl2002}     & Soultz, France                         \\
		\midrule
		\multirow{3}{*}{Nonlinear diffusion} & \cite{ShapiroDinske2009}   & Barnett shale   & 1/3                  \\
		                                     & \cite{HummelMller2009}     & Fenton Hill     & 1/2                  \\
		                                     & \cite{JohannEtAl2016}      & Multiple sites  & 1/2--1/3             \\
		\midrule
		\multirow{4}{*}{Aseismic slip}       & \cite{DuboeufEtAl2017}     & Rustrel, France & 0                    \\
		                                     & \cite{DeBarrosEtAl2021}    & Multiple sites  & 0.7--1               \\
		                                     & \cite{SaezUribe2023}       & Multiple sites  & 1                    \\
		                                     & \cite{DanreEtAl2024}       & Multiple sites  & 1                    \\
		\midrule
		Poroelasticity                       & \cite{GoebelEtAl2017}      & Oklahoma, USA   & 0                    \\
		\midrule
		Coulomb stress transfer              & \cite{YeoEtAl2020}         & Pohang, Korea                          \\
		\bottomrule
	\end{tabular}
\end{table}

As discussed previously, nonlinear diffusion during fluid injection can alter seismicity migration behavior. Studies have examined this process primarily in the context of hydraulic fractures. Additionally, no discussion has been made on its role in the quasi-linear migration of induced seismicity sequences. Nevertheless, a semi-analytical study of nonlinear diffusion has shown that pressure diffusion in rock fractures can result in a time exponent of 4/5 \cite{MurphyEtAl2004}. In this context, we aim to use well-constrained fault physical parameters to simulate fluid injection into a deformable fault under various injection boundary conditions and to evaluate the resulting impact on seismicity migration through comparison with observations.


%%%%%%%%%%%%%%%%%%%%%%%%%%%%%%%%%%%%%%%%%%%%%%%
%
% DATA SECTION and ACKNOWLEDGMENTS
%
%%%%%%%%%%%%%%%%%%%%%%%%%%%%%%%%%%%%%%%%%%%%%%%

\section*{Open Research Section}

\acknowledgments
This work of the Interdisciplinary Thematic Institut Geosciences for the energy system transition, as part of the ITI 2021-2028 program of the University of Strasbourg, CNRS and Inserm, was supported by Idex Unistra (ANR-10-IDEX-0002), and by SFRI-STRAT'US project (ANR-20-SFRI-0012).

\bibliography{references}

\end{document}
